\documentclass[12pt, a4paper]{article}
\usepackage[top=2cm, bottom=2cm, left=2cm, right=2cm]{geometry}
\usepackage[utf8]{inputenc}
\usepackage[T1]{fontenc}
\usepackage{array}
\usepackage{xcolor}
\usepackage{enumitem}
\usepackage[ngerman]{babel}
\usepackage{framed}
\usepackage{tgbonum}

% Define colors
\definecolor{headercolor}{RGB}{52, 152, 219}
\definecolor{examplecolor}{RGB}{46, 204, 113}
\definecolor{sectioncolor}{RGB}{41, 128, 185}

% Define environments
\newenvironment{taskbox}{%
    \begin{framed}
        \color{black}
    }{%
    \end{framed}
}

\newenvironment{examplebox}{%
    \begin{framed}
        \setlength{\fboxsep}{10pt}
        \color{black}
        \itshape
    }{%
    \end{framed}
}

% Define checkbox symbol
\newcommand{\checkbox}{\framebox[1em]{\phantom{X}}}

\setlength{\parindent}{0pt}
\renewcommand{\arraystretch}{1.5}

\begin{document}
    \begin{center}
        \vspace{0.3cm}
        {\color{headercolor}\Huge\textbf{Wir sind Kinderrechtsexperten!}}\\[0.5cm]
        \textbf{Name: \rule{4cm}{0.4pt} \hspace{1cm} Team: \rule{4cm}{0.4pt}}
    \end{center}

    \vspace{0.3cm}

    \section*{\large\color{headercolor}Auftrag:}
    
    \begin{taskbox}
        Ihr seid ein Team von Kinderrechtsexperten und sollt Regeln entwickeln, die arbeitende Kinder schützen.
    \end{taskbox}

    \vspace{0.3cm}

    \section*{\large\color{headercolor}Schritt 1: Wählt 3 Schutzbereiche aus}
    
    \begin{taskbox}
        \begin{tabular}{ll}
            \checkbox & Alter (Ab wann dürfen Kinder arbeiten?) \\
            \checkbox & Zeit (Wie lange und wann dürfen Kinder arbeiten?) \\
            \checkbox & Tätigkeiten (Welche Arbeit dürfen Kinder machen oder nicht machen?) \\
            \checkbox & Gesundheit (Wie schützen wir die Gesundheit der Kinder?) \\
            \checkbox & Bildung (Wie stellen wir sicher, dass Kinder zur Schule gehen können?) \\
            \checkbox & Freizeit (Wie stellen wir sicher, dass Kinder noch spielen können?) \\
        \end{tabular}
    \end{taskbox}

    \vspace{0.3cm}

    \section*{\large\color{headercolor}Schritt 2: Entwickelt für jeden Bereich 1-2 Regeln}
    
    \begin{examplebox}
        \textbf{Beispiel:}\\
        Bereich: \textbf{Gesundheit}\\
        Regel: \textit{Kinder dürfen nicht mit giftigen Stoffen arbeiten.}\\
        Begründung: \textit{Giftige Stoffe können Kinder krank machen und ihrem Körper, der noch wächst, schaden.}
    \end{examplebox}

    \vspace{0.3cm}

    \textbf{\color{sectioncolor}Euer 1. Schutzbereich:} \rule{8cm}{0.4pt}

    \begin{taskbox}
        Regel 1: \rule{\linewidth}{0.4pt}

        Begründung: \rule{\linewidth}{0.4pt}
        \rule{\linewidth}{0.4pt}

        \vspace{0.5cm}

        Regel 2: \rule{\linewidth}{0.4pt}

        Begründung: \rule{\linewidth}{0.4pt}
        \rule{\linewidth}{0.4pt}
    \end{taskbox}

    \vspace{0.3cm}

    \textbf{\color{sectioncolor}Euer 2. Schutzbereich:} \rule{8cm}{0.4pt}

    \begin{taskbox}
        Regel 1: \rule{\linewidth}{0.4pt}

        Begründung: \rule{\linewidth}{0.4pt}
        \rule{\linewidth}{0.4pt}

        \vspace{0.5cm}

        Regel 2: \rule{\linewidth}{0.4pt}

        Begründung: \rule{\linewidth}{0.4pt}
        \rule{\linewidth}{0.4pt}
    \end{taskbox}

    \textbf{\color{sectioncolor}Euer 3. Schutzbereich:} \rule{8cm}{0.4pt}

    \begin{taskbox}
        Regel 1: \rule{\linewidth}{0.4pt}

        Begründung: \rule{\linewidth}{0.4pt}
        \rule{\linewidth}{0.4pt}

        \vspace{0.5cm}

        Regel 2: \rule{\linewidth}{0.4pt}

        Begründung: \rule{\linewidth}{0.4pt}
        \rule{\linewidth}{0.4pt}
    \end{taskbox}

    \vspace{0.3cm}

    \section*{\large\color{headercolor}Schritt 3: Wählt eure wichtigste Regel aus}
    
    \begin{taskbox}
        Welche eurer Regeln findet ihr am wichtigsten? Unterstreicht sie und bereitet euch vor, sie der Klasse vorzustellen!

        \vspace{0.3cm}
        \textbf{Hilfen für Begründungen:}
        \begin{itemize}
            \item "Diese Regel ist wichtig, weil..."
            \item "Kinder brauchen diese Regel, damit..."
            \item "Ohne diese Regel könnte es passieren, dass..."
            \item "Wir müssen Kinder schützen vor..."
        \end{itemize}
    \end{taskbox}

    \vspace{0.5cm}
    \centering\large\textbf{Zeit: 15 Minuten | Präsentation: 1 Minute pro Gruppe}
\end{document}
