\documentclass[12pt, a4paper]{article}
\usepackage[top=2cm, bottom=2cm, left=2cm, right=2cm]{geometry}
\usepackage[utf8]{inputenc}
\usepackage[T1]{fontenc}
\usepackage{marvosym}
\usepackage{array}
\usepackage{xcolor}
\usepackage{enumitem}
\usepackage{amssymb}
\usepackage[ngerman]{babel}
\usepackage{wasysym}
\usepackage{framed}
\usepackage{tgbonum}

% Define colors
\definecolor{taskblue}{RGB}{0,127,175}
\definecolor{taskgreen}{RGB}{124,179,66}
\definecolor{taskbeige}{RGB}{237,201,175}
\definecolor{taskgold}{RGB}{212,175,55}
\definecolor{acceptgreen}{RGB}{0,150,0}
\definecolor{rejectred}{RGB}{200,0,0}

% Define environments
\newenvironment{taskbox}{%
	\begin{framed}
		\color{black}
	}{%
	\end{framed}
}

\newenvironment{infobox}{%
	\begin{framed}
		\color{black}
	}{%
	\end{framed}
}

\newenvironment{helpbox}{%
	\begin{framed}
		\setlength{\fboxsep}{10pt}
		\begin{minipage}{\dimexpr\textwidth-2\fboxsep\relax}
			\color{black}
			\setlength{\parindent}{0pt}
			\setlength{\parskip}{0.5em}
		}{%
		\end{minipage}
	\end{framed}
}

% Define checkbox symbol
\newcommand{\checkbox}{\raisebox{-0.2ex}{\Large$\square$}}

\setlength{\parindent}{0pt}
\renewcommand{\arraystretch}{1.5}

% Define missing symbols if not available in marvosym
\providecommand{\MVCheckmark}{\checkmark}
\providecommand{\MVCross}{\texttimes}
\providecommand{\Square}{\square}

\begin{document}
	\begin{center}
		\vspace{0.3cm}
		{\color{taskblue}\Huge\textbf{Detektivauftrag für Kinderrechte}}\\[0.3cm]
		{\Large\textbf{Fallbeispiel: Tom (12 Jahre)}}\\[0.5cm]
		\textbf{Name: \rule{4cm}{0.4pt} \hspace{1cm} Klasse: \rule{2cm}{0.4pt}}
	\end{center}
	
	\vspace{0.3cm}
	
	\section*{\color{taskgold}Deine Mission}
	
	\begin{infobox}
		\textit{„Hallo Rechtsdetektiv!\\
			Tom ist 12 Jahre alt und möchte sich etwas Taschengeld verdienen. Ein Bioladen in seiner Straße sucht Hilfe beim Einräumen von Regalen. Der Chef bietet an:}
		
		\begin{itemize}[leftmargin=*]
			\item \textbf{Arbeit:} Obst/Gemüse in Regale sortieren
			\item \textbf{Zeit:} Jeden Samstag von 8 bis 12 Uhr
			\item \textbf{Bezahlung:} 5 Euro pro Stunde
		\end{itemize}
		
		\textit{Deine Aufgabe: Überprüfe mit dem Gesetzes-Check, ob Tom diesen Job annehmen darf.}
	\end{infobox}
	
	\vspace{0.3cm}
	
	\section*{\color{taskgreen}Schritt 1: Fakten sammeln}
	
	\begin{taskbox}
		\begin{tabular}{|p{7.5cm}|p{7.5cm}|}
			\hline
			\textbf{Frage} & \textbf{Antwort (kurz)} \\
			\hline
			Wo arbeitet Tom? & \hrulefill \\
			\hline
			Was macht er? & \hrulefill \\
			\hline
			Wann und wie lange arbeitet er? & \hrulefill \\
			\hline
			Wie alt ist Tom? & \hrulefill \\
			\hline
		\end{tabular}
	\end{taskbox}
	
	\vspace{0.3cm}
	
	\pagebreak
	
	\section*{\color{taskblue}Schritt 2: Gesetzes-Check}
	
	\begin{taskbox}
		\textit{Prüfe jede Regel – trifft sie auf Tom zu?}
		
		\begin{tabular}{|p{4cm}|p{8cm}|p{3cm}|}
			\hline
			\textbf{Wo steht das Gesetz?} & \textbf{Was steht im Gesetz?} & \textbf{Stimmmt das bei Tom?} \\ 
			\hline
			JArbSchG §5 & \begin{minipage}[t]{8cm}
				\begin{itemize}[leftmargin=*,nosep]
					\item Kinder unter 13 Jahren dürfen nicht arbeiten
					\item Nur Zeitungsausträger ab 12 Jahren sind erlaubt
				\end{itemize}
			\end{minipage} & \checkbox \\ 
			\hline
			JArbSchG §7 & \begin{minipage}[t]{8cm}
				\begin{itemize}[leftmargin=*,nosep]
					\item 13-14 Jährige dürfen maximal 2 Stunden pro Tag arbeiten
					\item Arbeit nur an 5 Tagen pro Woche
					\item Nicht vor 6 Uhr morgens (vor der Schule)
				\end{itemize}
			\end{minipage} & \checkbox \\ 
			\hline
			JArbSchG §22 & \begin{minipage}[t]{8cm}
				\begin{itemize}[leftmargin=*,nosep]
					\item Keine gefährlichen Arbeiten (z.B. schwere Sachen tragen)
					\item Keine Arbeiten mit Maschinen
					\item Nur leichte Hilfsarbeiten sind erlaubt
				\end{itemize}
			\end{minipage} & \checkbox \\ 
			\hline
			MiLoG §1 & \begin{minipage}[t]{8cm}
				\begin{itemize}[leftmargin=*,nosep]
					\item Mindestlohn (12,41 €/h) erst ab 18 Jahren
					\item Unter 18 Jahren: Chef kann frei entscheiden, wie viel er zahlt
				\end{itemize}
			\end{minipage} & \checkbox \\ 
			\hline
		\end{tabular}
	\end{taskbox}
	
	\vspace{0.3cm}
	
	\section*{\color{taskgold}Schritt 3: Urteil fällen}
	
	\begin{taskbox}
		\textit{Setze das Puzzle zusammen!}
		
		\textbf{Tom darf diesen Job:}
		\begin{itemize}[leftmargin=*]
			\item \checkbox\ \textcolor{acceptgreen}{ANNEHMEN} (wenn alle Regeln eingehalten sind)
			\item \checkbox\ \textcolor{rejectred}{NICHT ANNEHMEN} (wenn eine oder mehrere Regeln verletzt sind)
		\end{itemize}
		
		\textbf{Begründung:} Unterstreiche die zutreffenden Gründe:\\
		1. [Alter zu niedrig] 2. [Arbeitszeit zu lang] 3. [Arbeit zu schwer] 4. [Lohn zu niedrig]
	\end{taskbox}
	
	\vspace{0.3cm}
	
	\section*{\color{taskgreen}Reflexionsfragen für Schnelldenker}
	
	\begin{taskbox}
		\begin{enumerate}
			\item Warum gibt es Altersgrenzen für Arbeit?
			\item Was könnte passieren, wenn Tom trotzdem arbeitet?
			\item Wie könnte Tom sein Taschengeld legal verdienen? (Tipp: Ab 13 erlaubt?)
		\end{enumerate}
	\end{taskbox}
	
\end{document}